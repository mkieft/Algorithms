\documentclass[12pt]{article}
\setlength{\oddsidemargin}{0in}
\setlength{\evensidemargin}{0in}
\setlength{\textwidth}{6.5in}
\setlength{\parindent}{0in}
\setlength{\parskip}{\baselineskip}
\usepackage{amsmath,amsfonts,amssymb}
\usepackage{graphicx}
\usepackage[]{algorithmicx}

\usepackage{fancyhdr}
\pagestyle{fancy}

%\usepackage{hyperref}


\setlength{\headsep}{36pt}

\begin{document}

\lhead{{\bf CSCI 3104, Algorithms \\ Explain-It-Back 9} }
\rhead{Name: \fbox{\phantom{This is a really long name}} \\ ID: \fbox{\phantom{This is a student ID}} \\ {\bf Profs.\ Grochow \& Layer\\ Spring 2019, CU-Boulder}}
\renewcommand{\headrulewidth}{0.5pt}

\phantom{Test}
A finance colleague asks for your help in developing software that will help
her automate some of the buy and sell orders that she receives. Simplifying
things a bit, she describes buy orders as target asset and a dollar amount to
spend and sell orders as target asset and an amount of the asset to sell. As
you develop this application you see a funny pattern. The US dollar (USD) to
Pound sterling rate is 0.77 (GBP), the GBP to Canadian dollar (CAD) rate it
1.75, and the CAD to USD rate is 0.75. You get very excited by this observation
and immediately stop work on the automated buy/sell tool and start implementing
a shortest path algorithm. After a few tests you are confident in your idea,
now you pitch this new method to your friend.


\pagebreak

\newpage
\mbox{}
\newpage
\pagebreak
\end{document}
